%% Requires compilation with XeLaTeX or LuaLaTeX
\documentclass[10pt,xcolor={table,dvipsnames},t]{beamer}
\usetheme{UCBerkeley}
\title[AP-SD-M]{Fixed-point Q-Learning}
\subtitle{Project Work on Digital Systems M}
\author{Kevin Michael Frick \and Davide Ragazzini}
\institute{University of Bologna}
\date{\today}

\begin{document}

\begin{frame}
  \titlepage
\end{frame}

\begin{frame}{Introduction}

\begin{itemize}
  \item Objective: implementing Q-value update, exploration etc. in fixed point
  \item We compare it to floating point in terms of performance and resource usage
  \item We also discovered a bug in our initial floating-point implementation, so we re-ran our tests to ensure a fair comparison
\end{itemize}

\end{frame}


\begin{frame}{Performance: periods to convergence}
\begin{table}[]
\begin{tabular}{llllll}
\textbf{Initalization of $Q$}  & \textbf{{[}0,1{]}} & \textbf{{[}1, 2{]}} & \textbf{{[}2, 3{]}} & \textbf{{[}3, 4{]}} & \textbf{{[}4, 5{]}} \\
\textbf{Fixed point}    & 234                & 192                 & 196                 & 200                 & 202                 \\
\textbf{Floating point} & 176                & 186                 & 191                 & 195                 & 199                
\end{tabular}
\end{table}

There is a slight fraction of tests that do not converge when initializing in $[0, 1]$ and using fixed point. This does not happen when using floating point.

We see that the two formulations are almost equivalent in terms of period to convergence.

\end{frame}


\begin{frame}{Resource usage}

We compare side-by-side the resource usage table given by Vitis HLS.

\begin{table}[]
\begin{tabular}{lll}
               & \textbf{Floating point} & \textbf{Fixed point} \\
\textbf{SLICE} & 1579                    & 1778                 \\
\textbf{LUT}   & 4416                    & 5271                 \\
\textbf{FF}    & 4224                    & 4456                 \\
\textbf{DSP}   & 19                      & 31                   \\
\textbf{BRAM}  & 6                       & 6                    \\
\textbf{SRL}   & 90                      & 118                 
\end{tabular}
\end{table}

\end{frame}

\begin{frame}{Timing}
\begin{table}[]
\begin{tabular}{lll}
                                & \textbf{Floating point} & \textbf{Fixed point} \\
\textbf{CP post-synthesis}      & 8.276ms                   & 8.276ms                \\
\textbf{CP post-implementation} & 9.149ms                 & 9.285ms \\
\textbf{Time per iteration} & 77ms & 63ms
\end{tabular}
\end{table}

Critical paths maintain basically the same speed, but fixed point enjoys a faster overall execution time.
\end{frame}

\begin{frame}{Conclusions}
    
    \begin{itemize}
        \item There is no improvement in performance
        \item Floating-point uses less resources
        \item RL is an inherently floating-point task, and the best fixed-point can do is optimize for execution time
        \item This is probably due to the DSPs in more modern FPGAs. Older, slower ones would see fixed-point fare better
    \end{itemize}
    
\end{frame}

\end{document}

