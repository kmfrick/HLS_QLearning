\documentclass{beamer}
\usepackage{amsmath}
\usepackage[english]{babel}
\usepackage{commath}
\usepackage[autostyle, english = american]{csquotes}
\usepackage{graphicx}
\usepackage{hyperref}
\usepackage{cleveref}
\usepackage[sort,round]{natbib}
\usepackage{parskip}
\usepackage{subcaption}
\usepackage{microtype}
\usepackage[utf8]{microtype}
\usetheme{Boadilla}
\begin{document}
\title{Parallel Q-Learning on FPGA}
\author{Kevin Michael Frick \and Davide Ragazzini}
\institute{Università di Bologna}
\date{\today}
\setbeamercovered{transparent}
\begin{frame}
	\titlepage
\end{frame}

\begin{frame}
	\frametitle{Q-Learning}
	\begin{itemize}
		\item \emph{On-policy}, \emph{classical} reinforcement learning algorithm
	\item Used to estimate an optimal policy in Markov decision processes, i.e. the policy that maximizes the expected value of the total reward over all successive steps
	\item The \emph{reward function} $Q(s, a)$ is approximated using a state-action matrix known as the Q-table
	\item After taking action $a$ in state $s$, obtaining reward $r$, $Q[s][a]$ is updated following $Q[s][a] += \alpha  (r + \gamma  \bar{Q}[s] - Q[s][a])$
	\end{itemize}
	{
	
	\centering

	%\includegraphics[width=0.5\textwidth]{neural_net2.jpg}


	}
\end{frame}

\begin{frame}[allowframebreaks]
	\frametitle{References}
	\bibliographystyle{plainnat}
	\nocite{*}
	\bibliography{main}
\end{frame}
\end{document}
